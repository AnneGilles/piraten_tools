%
% 	Einsteigerworkshop-Praesentation "Piraten-Tools"
%
%	Christoph Scheid
%
% 	to produce verbose slides (one page per item/bullet point), use
\documentclass{beamer}
%
% 	but if you want to produce handouts for printing, use
%\documentclass[handout]{beamer} 
%
%	and while developing, use this for speed :-)
%\documentclass[draft,handout]{beamer}% 		no graphics (just placeholders), one slide per frame
%\documentclass[draft]{beamer}% 				no graphics (just placeholders), one slide per item
%
%	make graphics optional in handout mode (preparation)
\usepackage{ifthen}
\newboolean{makeHandoutWithoutGraphics}
%
%	uncomment this for handout mode WITHOUT graphics
%      comment out for verbose slides WITH images
%\setboolean{makeHandoutWithoutGraphics}{true}
%

%
% \usepackage{beamerthemesplit} // Activate for custom appearance
%
%	textpos is used to place images in arbitrary positions
\usepackage[absolute, overlay]{textpos}
\setlength{\TPHorizModule}{10mm}
\setlength{\TPVertModule}{\TPHorizModule}
\textblockorigin{0mm}{0mm}

\usepackage{geometry}
\usepackage{texments} %Note that if you use the texments commands 
% to highlight text inside of beamer presentations, you need to 
% mark the frames with the code “fragile”, like this:
% \begin{frame}[fragile] ... \end{frame}

\usepackage{color}
%\usepackage{movie15} % embed videos; playback only works in acrobat
\usepackage{listings} \lstset{numbers=left, numberstyle=\tiny, numbersep=5pt} \lstset{language=Python}
\usepackage{hyperref}
\vfill
\title{piratiger Einsteigerworkshop}
\author{Christoph Scheid}
%\date{\today}


\usetheme{Singapore} % this one takes time when compiling,
      % while it generates a clickable overview on each page
%\usetheme{Bergen}

\begin{document}

% for hyperref: frenchlinks=true ?
\hypersetup{frenchlinks=true}

\AtBeginSection[]
{
   \begin{frame}
       \frametitle{\"Ubersicht}
       \tableofcontents[currentsection]
   \end{frame}
}

%% Outline slides at beginning of subsections?
%\AtBeginSubsection[]
%{
%   \begin{frame}
%       \frametitle{\"Ubersicht}
%       \tableofcontents[currentsection,currentsubsection]
%   \end{frame}
%}





\frame{
\frametitle{piratiger Einsteigerworkshop}

	\begin{textblock}{1}(5,3.5)
     		\includegraphics<1>[scale=.2]{img/podium}
   	\end{textblock}

	~\\ \vfill % push stuff down a bit
	
	\huge 
	Werkzeuge,\\ 
	Techniken,\\
	Tipps, ...\\
	\normalsize
	
	\vfill
	Christoph Scheid\\
	03.06.2012
}

\frame{
\frametitle{Inhalt}
\tableofcontents
}

%%%%%%%%%%%%%%%%%%%%%%%%%%%%%%%%%%%%%%%%%%%%%%%%%%%%%%%
%%%%%%%%%%%%%%%%%%%%%%%%%%%%%%%%%%%%%%%%%%%%%%%%%%%%%%%
%%%%%%%%%%%%%%%%%%%%%%%%%%%%%%%%%%%%%%%%%%%%%%%%%%%%%%%
\section[Motivation]{Motivation}
%%%%%%%%%%%%%%%%%%%%%%%%%%%%%%%%%%%%%%%%%%%%%%%%%%%%%%%
\frame{
\huge{The medium is the massage}

\vspace{2em}
\hfill\normalsize
Marshall McLuhan
}
%%%%%%%%%%%%%%%%%%%%%%%%%%%%%%%%%%%%%%%%%%%%%%%%%%%%%%%
%%%%%%%%%%%%%%%%%%%%%%%%%%%%%%%%%%%%%%%%%%%%%%%%%%%%%%%
\subsection{Kategorien}
%%%%%%%%%%%%%%%%%%%%%%%%%%%%%%%%%%%%%%%%%%%%%%%%%%%%%%%

\frame{
\frametitle{Tools -- Kategorien}
\begin{itemize}
	\item<1-> Sprachvariet\"at
	{\begin{itemize}
		\item<2-> Schriftlich	
		\item<3-> Chat \hfill  ;-) 
		\item<2-> M\"undlich
	\end{itemize}}
	\item<1-> Richtung
	{\begin{enumerate}
		\item<4-> Intern	
		\item<4-> Extern
	\end{enumerate}}
	\item<1-> Persistenz
	{\begin{enumerate}
		\item<5-> fl\"uchtig
		\item<5-> stabil
		\item<5-> unl\"oschbar		
	\end{enumerate}}
\end{itemize}
}


\frame{
\frametitle{Tools -- Kategorien II}
\transboxin<2>
\begin{itemize}
			{\begin{textblock}{0}(5,3.5)
     				\includegraphics<3>[trim=1cm 3cm 1cm 3cm,clip=true,scale=.2]{img/one-to-many}
   			\end{textblock}}

	\item<1->Anzahl{
	\begin{itemize}
		\item<2-> One-to-One
		\item<3-> One-to-Many
		\item<4-> Many-to-Many
	\end{itemize}
	}
	\item<1-> Zeitlich{
		\begin{itemize}
		\item<5-> synchron
		\item<5-> asynchron
		\end{itemize}
	}
\end{itemize}
}

%%%%%%%%%%%%%%%%%%%%%%%%%%%%%%%%%%%%%%%%%%%%%%%%%%%%%%%
%\subsection{Toolliste}

\frame{
\frametitle{Tools -- Liste}
\begin{itemize}
	\item<1-> Webseite 		\hfill \small\url{http://piratenpartei-marburg.de}
	\item<2-> Wiki 			\hfill \small\url{https://wiki.piratenpartei.de}
	\item<3-> Mailingliste 	\hfill\small\url{marburg@piratenpartei-hessen.de}
	\item<4-> Sync-Forum 	\hfill\small\url{https://news.piratenpartei.de} 
	\item<5-> Pad  			\hfill\small\url{https://marburg.piratenpad.de}
	\item<6-> LiquidFeedback  \hfill\small\url{https://lqfb.piratenpartei.de}
	\item<7-> vMB 			\hfill\small\url{https://vote-mabi.piratenpartei-hessen.de}
	\item<8-> Telko  		\hfill\tiny\url{https://wiki.piratenpartei.de/HE:Telefonkonferenz}
	\item<9-> Mumble 		\hfill\small\url{https://wiki.piratenpartei.de/Mumble}
	\item<10-> IRC 		\hfill \href{http://www.piratenpartei-marburg.de/mitmachen/irc/webchat}{\#piratenpartei-marburg} auf \href{http://www.piratenpartei-marburg.de/mitmachen/irc/webchat}{freenode} 
	\item<11-> Chat \hfill Jabber, ICQ*, Skype*, \dots
\end{itemize}
}

%%%%%%%%%%%%%%%%%%%%%%%%%%%%%%%%%%%%%%%%%%%%%%%%%%%%%%%
%%%%%%%%%%%%%%%%%%%%%%%%%%%%%%%%%%%%%%%%%%%%%%%%%%%%%%%
%%%%%%%%%%%%%%%%%%%%%%%%%%%%%%%%%%%%%%%%%%%%%%%%%%%%%%%
\section{Tools}

%%%%%%%%%%%%%%%%%%%%%%%%%%%%%%%%%%%%%%%%%%%%%%%%%%%%%%%%
%\subsection{}
%\frame{%
%\frametitle{Tools}%
%  \begin{textblock}{0}(1,1.5)%
%    \includegraphics<1>[scale=.8]{img/Logo-Oeffentlichkeitsarbeit}%
%  \end{textblock}%
%
%}

%%%%%%%%%%%%%%%%%%%%%%%%%%%%%%%%%%%%%%%%%%%%%%%%%%%%%%%
\subsection{Vorstellung}
%%%%%%%%%%%%%%%%%%%%%%%%%%%%%%%%%%%%%%%%%%%%%%%%%%%%%%%
\frame{
\frametitle{Webseite}
Aussendarstellung, Information der \"Offentlichkeit
\begin{itemize}
	\item<1-> News
	\item<2-> Einladungen
	\item<3-> Pressemitteilungen
	\item<4-> Blogs
	\item<5-> Spam (ToDo: Screenshot v. wegklicken)
%	\item<6-> 
%	\item<7-> 
\end{itemize}
}


%% %%%%%%%%%%%%%%%%%%%%%%%%%%%%%%%%%%%%%%%%%%%%%%%%%%%%%%%
%\subsection{Wiki}
%% %%%%%%%%%%%%%%%%%%%%%%%%%%%%%%%%%%%%%%%%%%%%%%%%%%%%%%%
\frame{   
  \frametitle{Wiki}

	Texte \it im Wiki verstecken\rm
  \begin{itemize}
    \item <1-> Speichern
    \item <2-> Versionieren/Vergleichen
    \item <3-> eigene Benutzerseite:  \href{https://wiki.piratenpartei.de/Benutzer:AnneGilles}{/Benutzer:AnneGilles}
    \item <4-> Editierhilfe
    \item <5-> Formatierungshilfen
    \item <6-> KV MR Namensraum: \href{https://wiki.piratenpartei.de/HE:Marburg}{/HE:Marburg}
%%     \item <7-> 
%%     \item <8-> 
  \end{itemize}
}

%%%%%%%%%%%%%%%%%%%%%%%%%%%%%%%%%%%%%%%%%%%%%%%%%%%%%%%
%\subsection{Etherpad}
%%%%%%%%%%%%%%%%%%%%%%%%%%%%%%%%%%%%%%%%%%%%%%%%%%%%%%%
\frame{
\frametitle{Etherpad}
Gemeinsam Texte schreiben -- Gleichzeitig!
\begin{itemize}
	\item<1-> Prima f\"ur Protokolle, Brainstorming, Texte, ...\ifthenelse{\boolean{makeHandoutWithoutGraphics}}{}{\begin{textblock}{0}(0.5,4.5)
     			\includegraphics<1>[scale=.3]{img/pad-2}
   			\end{textblock}}
	\item<2-> Sehen, wer was schreibt\ifthenelse{\boolean{makeHandoutWithoutGraphics}}{}{\begin{textblock}{0}(7,4.5)
     			\includegraphics<2>[scale=.5]{img/pad-authors}
   			\end{textblock}}
	\item<3-> Versionierung (*)
	\item<4-> Versionen markieren/speichern\ifthenelse{\boolean{makeHandoutWithoutGraphics}}{}{\begin{textblock}{0}(0.5,6.5)
     			\includegraphics<4>[scale=.3]{img/pad-revisions}
   			\end{textblock}}
	\item<5-> Time Slider
	\item<6-> Export\ifthenelse{\boolean{makeHandoutWithoutGraphics}}{}{\begin{textblock}{0}(8,4.5)
     			\includegraphics<6>[scale=.4]{img/pad-export}
   			\end{textblock}}
	\item<7-> Gruppenpads (anmelden f\"ur \"Ubersicht){
	\begin{itemize}
		\item \url{https://marburg.piratenpad.de}
		\item gehen nicht verloren
		\item Links auf der Webseite{\begin{textblock}{0}(9.5,1.2)
     			\includegraphics<7>[scale=.38]{img/pad-drupal}
   			\end{textblock}}
	\end{itemize}
	}
	\item<8-> Neues Pad anlegen

\end{itemize}
}


%% %%%%%%%%%%%%%%%%%%%%%%%%%%%%%%%%%%%%%%%%%%%%%%%%%%%%%%%
%\subsection{Mailingliste}
%% %%%%%%%%%%%%%%%%%%%%%%%%%%%%%%%%%%%%%%%%%%%%%%%%%%%%%%%
\frame{   
  \frametitle{Mailingliste}
\url{mailto:marburg@piratenpartei-hessen.de}
  \begin{itemize}
    \item <1-> Emails werden an alle verteilt 
    \item <2-> z.\,Zt.\,105 Abonnenten (offene Liste)
    \item <3-> Moderation (Gr\"o\ss{}e, Absender, Adressaten)
    \item <4-> keine Anh\"ange !
    \item <5-> besser: Upload, URL senden
    \item <6-> Zitierstil (ToDo: Beispiel)
    \item <7-> Ordner anlegen, Filter anlegen (ToDo: Beispiel)
%%     \item <8-> 
   \end{itemize}
}



%% %%%%%%%%%%%%%%%%%%%%%%%%%%%%%%%%%%%%%%%%%%%%%%%%%%%%%%%
%\subsection{Sync-Forum}
%% %%%%%%%%%%%%%%%%%%%%%%%%%%%%%%%%%%%%%%%%%%%%%%%%%%%%%%%
 \frame{   
   \frametitle{Sync-Forum}

	Mailingliste(n) lesbar als Forum\\
	\small\url{http://news.piratenpartei.de/forumdisplay.php?fid=329}

   \begin{itemize}
     \item <1-> Marburger ML lesen\ifthenelse{\boolean{makeHandoutWithoutGraphics}}{}{\begin{textblock}{0}(1,2.1)
     			\includegraphics<1>[scale=.27]{img/sync-forum-1}
   			\end{textblock}}
     \item <2-> keine X Emails empfangen m\"ussen
     \item <3-> alle* anderen mitlesen \hfill *fast
     \item <4-> auch Antworten schreiben!
     \item <5-> Antworten werden als Email verteilt
     \item <6-> Zitieren schwierig
%     \item <7-> 
%     \item <8-> 
   \end{itemize}
 }
 
 
%% %%%%%%%%%%%%%%%%%%%%%%%%%%%%%%%%%%%%%%%%%%%%%%%%%%%%%%%
%\subsection{Chat}
%% %%%%%%%%%%%%%%%%%%%%%%%%%%%%%%%%%%%%%%%%%%%%%%%%%%%%%%%
 \frame{   
   \frametitle{Chat}

	Direktnachrichten bilateral

   \begin{itemize}
     \item <1-> Jabber empfohlen{\begin{textblock}{0}(8,2.1)
     			\includegraphics<1>[scale=.27]{img/jabber}
   			\end{textblock}}
     \item <2-> wie Telefon ohne Ton
     \item <4-> OTR-plugin: Verschl\"usselung\ifthenelse{\boolean{makeHandoutWithoutGraphics}}{}{\begin{textblock}{0}(6,3)
     			\includegraphics<3>[scale=.2]{img/jabber-talk}
   			\end{textblock}}
%     \item <4-> 
%     \item <5-> 
%     \item <6-> 
%     \item <7-> 
%     \item <8-> 
   \end{itemize}
 }



%% %%%%%%%%%%%%%%%%%%%%%%%%%%%%%%%%%%%%%%%%%%%%%%%%%%%%%%%
%\subsection{IRC}
%% %%%%%%%%%%%%%%%%%%%%%%%%%%%%%%%%%%%%%%%%%%%%%%%%%%%%%%%
 \frame{   
   \frametitle{IRC}

	Direktnachrichten multilateral

   \begin{itemize}
     \item <1-> Marburger Channel
     			{\begin{textblock}{0}(7,1.4)
     			\includegraphics<1>[scale=.2]{img/irc-webchat-01}
   			\end{textblock}}
     \item <2-> St\"andiger Parteitag!\ifthenelse{\boolean{makeHandoutWithoutGraphics}}{}
          			{\begin{textblock}{0}(7,1.4)
     			\includegraphics<2>[scale=.2]{img/irc-webchat-02}
   			\end{textblock}}

     \item <3-> ohne GO-Gymnastik\ifthenelse{\boolean{makeHandoutWithoutGraphics}}{}
     			{\begin{textblock}{0}(7,1.4)
	     			\includegraphics<3>[scale=.4]{img/irc-webchat-03}
   			\end{textblock}}
     			
     \item <4-> WebChat{
     	\begin{itemize} 
     		\item \tiny\href{http://www.piratenpartei-marburg.de/mitmachen/irc/webchat}{http://www.piratenpartei-marburg.de} \normalsize\href{http://www.piratenpartei-marburg.de/mitmachen/irc/webchat}{/mitmachen/irc/webchat}
     	\end{itemize}}

			\ifthenelse{\boolean{makeHandoutWithoutGraphics}}{}{\begin{textblock}{0}(1,2.1)
	     			\includegraphics<4>[scale=.4]{img/irc-webchat-04}
   			\end{textblock}}

     			\ifthenelse{\boolean{makeHandoutWithoutGraphics}}{}{\begin{textblock}{0}(1,2.1)
     				\includegraphics<5>[scale=.35]{img/irc-webchat-05}
   			\end{textblock}}

     			\ifthenelse{\boolean{makeHandoutWithoutGraphics}}{}{\begin{textblock}{0}(1,2.1)
     			\includegraphics<6>[scale=.35]{img/irc-webchat-06}
   			\end{textblock}}

     			\ifthenelse{\boolean{makeHandoutWithoutGraphics}}{}{\begin{textblock}{0}(1,2.1)
     			\includegraphics<7>[scale=.35]{img/irc-webchat-07}
   			\end{textblock}}

     \item <8-> Anstupsen durch Erw\"ahnen eines Namens
     			\ifthenelse{\boolean{makeHandoutWithoutGraphics}}{}{\begin{textblock}{0}(2,7)
     			\includegraphics<8>[scale=1]{img/irc-webchat-08}
   			\end{textblock}}
      
     			\ifthenelse{\boolean{makeHandoutWithoutGraphics}}{}{\begin{textblock}{0}(1,2.1)
     			\includegraphics<9>[scale=.35]{img/irc-webchat-09}
   			\end{textblock}}
	\item<10> viiiiiiele andere Kan\"ale{
	     	\begin{itemize} 
     			\item \#piraten-hessen
     			\item \#piratenpartei
     			\item \#piraten-...
	     	\end{itemize}}
     			\ifthenelse{\boolean{makeHandoutWithoutGraphics}}{}{\begin{textblock}{0}(0,0)
     			\includegraphics<11>[scale=.35]{img/irc-channel-list}
   			\end{textblock}}

%     \item <6-> 
%     \item <7-> 
%     \item <8-> 
   \end{itemize}
 }

%% %%%%%%%%%%%%%%%%%%%%%%%%%%%%%%%%%%%%%%%%%%%%%%%%%%%%%%%
%\subsection{virtuelles Meinungsbild}
%% %%%%%%%%%%%%%%%%%%%%%%%%%%%%%%%%%%%%%%%%%%%%%%%%%%%%%%%
 \frame{   
   \frametitle{virtuelles Meinungsbild (vMB)}



   \begin{itemize}
   	\item<1,4->Abstimmungen zu bestimmten Themen
		   \ifthenelse{\boolean{makeHandoutWithoutGraphics}}{}{\begin{textblock}{0}(1,4)
     			\includegraphics<1>[scale=.3]{img/vMB-list}% https://vote-mabi.piratenpartei-hessen.de/
   			\end{textblock}}
     		\ifthenelse{\boolean{makeHandoutWithoutGraphics}}{}{\begin{textblock}{0}(2,2)
     			\includegraphics<2>[scale=.3]{img/vMB-text}% https://vote-mabi.piratenpartei-hessen.de/auswertung.php?id=75728&lang=de-informal
   			\end{textblock}}
	     	\ifthenelse{\boolean{makeHandoutWithoutGraphics}}{}{\begin{textblock}{0}(2,2)
     			\includegraphics<3>[scale=.3]{img/vMB-result}% https://vote-mabi.piratenpartei-hessen.de/auswertung.php?id=75728&lang=de-informal
   			\end{textblock}}
     	\item <4-> getriggert per Antrag an den Vorstand

     	\item <4-> einreichen: Text + Antwortoptionen
          	
     	\item <5-> Emails mit Token gehen raus
	\item<6-> nur f\"ur Piraten$^{TM}$
	\item<7-> sich umentscheiden
	\item<8-> Ergebniss nachvollziehbar \ifthenelse{\boolean{makeHandoutWithoutGraphics}}{}{\begin{textblock}{0}(6,6)
     			\includegraphics<8>[scale=.25]{img/vMB-token}% https://vote-mabi.piratenpartei-hessen.de/statistiken/75728.token.html
   			\end{textblock}}
			% in handout-mode: ergebnisgrafik einblenden
			\ifthenelse{\boolean{makeHandoutWithoutGraphics}}{\begin{textblock}{0}(6.8, 5.3)
     			\includegraphics<2>[scale=.18]{img/vMB-result}% https://vote-mabi.piratenpartei-hessen.de/auswertung.php?id=75728&lang=de-informal
   			\end{textblock}}{}


%     \item <4-> 
%     \item <5-> 
%     \item <6-> 
%     \item <7-> 
%     \item <8-> 
   \end{itemize}
 }



%% %%%%%%%%%%%%%%%%%%%%%%%%%%%%%%%%%%%%%%%%%%%%%%%%%%%%%%%
%% \frame{
%% \frametitle{}
%%   \begin{textblock}{10}(1,2)
%%     %	  \includegraphics<1>[scale=.42]{}
%%   \end{textblock}%
%% }



%% %%%%%%%%%%%%%%%%%%%%%%%%%%%%%%%%%%%%%%%%%%%%%%%%%%%%%%%
%% %%%%%%%%%%%%%%%%%%%%%%%%%%%%%%%%%%%%%%%%%%%%%%%%%%%%%%%
%%%%%%%%%%%%%%%%%%%%%%%%%%%%%%%%%%%%%%%%%%%%%%%%%%%%%%%%%
 \section{Vergleiche}
%% %%%%%%%%%%%%%%%%%%%%%%%%%%%%%%%%%%%%%%%%%%%%%%%%%%%%%%%
%%%%%%%%%%%%%%%%%%%%%%%%%%%%%%%%%%%%%%%%%%%%%%%%%%%%%%%
\subsection{Vergleiche -- was taugt f\"ur was?}
%%%%%%%%%%%%%%%%%%%%%%%%%%%%%%%%%%%%%%%%%%%%%%%%%%%%%%%
 \frame{
 \frametitle{Vergleiche}
%   \begin{textblock}{0}(5,3.5)
%     \includegraphics<1>[scale=.5]{pics/Selenium-big-logo}
 %  \end{textblock}%
   \begin{itemize}
     \item <1-> Telko vs. Mumble
     \item <2-> Telefon vs. SIP
     \item <3-> Etherpad vs. Wiki
     \item <4-> Mailingliste vs. Sync-Forum
     \item <5-> Zitieren: Inline vs. ToFu vs. \dots 
   \end{itemize}
 }

%% %%%%%%%%%%%%%%%%%%%%%%%%%%%%%%%%%%%%%%%%%%%%%%%%%%%%%%%
\frame{
\frametitle{Telko vs. Mumble}
    
    \small Telefonkonferenzen sind Treffen in virtuellen R\"aumen. \\ \normalsize

%~ \\
~ \\
 ~ \\ \normalsize


~ \hfill
\begin{tabular}{ l  c  c  c }
  \hline                       
  typ       		& Telko 	& SIP 	& Mumble \\ \hline
  Telefon? 	&       + 	&      -     	&    -    \\
  Computer?  	&      ? 	&      +  	&    +   \\
  \hline  
\end{tabular} \hfill ~ \\
~ \\
~\\
~ \hfill \small{\url{http://wiki.piratenpartei.de/HE:Telefonkonferenz}} \\

}


 %%%%%%%%%%%%%%%%%%%%%%%%%%%%%%%%%%%%%%%%%%%%%%%%%%%%%%%
 \frame{   
   \frametitle{Telefon vs. SIP vs. Stream}

   \begin{itemize}
     \item <1-> freihand 
     \item <2-> stumm stellen '*1'
     \item <3-> laut h\"oren (nur wenn stumm)
     \item <4-> SIP: kostenlos (z.B. sipgate.de)
     \item <5-> Stream{\begin{textblock}{0}(7.2,2.1)
     			\includegraphics<5>[trim=2cm 2cm 2cm 2cm,clip=true,scale=.2]{img/sip-1000}
   			\end{textblock}}
     \item <6-> \url{http://sip.piratenpartei-hessen.de:8000/raum1590.ogg}
%     \item <7-> 
%     \item <8-> 
   \end{itemize}
 }

 %%%%%%%%%%%%%%%%%%%%%%%%%%%%%%%%%%%%%%%%%%%%%%%%%%%%%%%
 \frame{   
   \frametitle{Etherpad vs. Wiki}

   \begin{itemize}
     \item <1-> gemeinsam
     \item <2-> gleichzeitig
     \item <3-> dauerhaft
     \item <4-> versionierung
%     \item <5-> 
%     \item <6-> 
%     \item <7-> 
%     \item <8-> 
   \end{itemize}
 }

 %%%%%%%%%%%%%%%%%%%%%%%%%%%%%%%%%%%%%%%%%%%%%%%%%%%%%%%
 \frame{   
   \frametitle{Mailingliste vs. Sync-Forum}

   \begin{itemize}
     \item <1-> zu viele emails?
     \item <2-> zitieren
     \item <3-> andere listen
     \item <4-> AGBs
%     \item <5-> 
%     \item <6-> 
%     \item <7-> 
%     \item <8-> 
   \end{itemize}
 }

 %%%%%%%%%%%%%%%%%%%%%%%%%%%%%%%%%%%%%%%%%%%%%%%%%%%%%%%
 \frame{   
   \frametitle{Inline vs. ToFu vs. \ldots}

   \begin{itemize}
     \item <1-> sauber zitieren
     \item <2-> auf das antworten, was gefragt war
     \item <3-> zu viel scrollen
%     \item <4-> 
%     \item <5-> 
%     \item <6-> 
%     \item <7-> 
%     \item <8-> 
   \end{itemize}
 }

%% %%%%%%%%%%%%%%%%%%%%%%%%%%%%%%%%%%%%%%%%%%%%%%%%%%%%%%%
%% %%%%%%%%%%%%%%%%%%%%%%%%%%%%%%%%%%%%%%%%%%%%%%%%%%%%%%%
%% %%%%%%%%%%%%%%%%%%%%%%%%%%%%%%%%%%%%%%%%%%%%%%%%%%%%%%%
\section{Workflows}
%% %%%%%%%%%%%%%%%%%%%%%%%%%%%%%%%%%%%%%%%%%%%%%%%%%%%%%%%
\frame{
\frametitle{Texte erstellen}

Anfragen, Antr\"age, Brainstorming, Pressemitteilung, Homepage-Text, \dots  
\begin{enumerate}
    \item <1-> in ein Pad (mit geeignetem Namen \& URL)
    \item <2-> Mailingliste einladen (mit URL)
    \item <3-> Telko zur Finalisierung
    \item <4-> Text ins Wiki \"ubernehmen, Pad aufr\"aumen   
\end{enumerate}
}

%% %%%%%%%%%%%%%%%%%%%%%%%%%%%%%%%%%%%%%%%%%%%%%%%%%%%%%%%
%% %%%%%%%%%%%%%%%%%%%%%%%%%%%%%%%%%%%%%%%%%%%%%%%%%%%%%%%
%% %%%%%%%%%%%%%%%%%%%%%%%%%%%%%%%%%%%%%%%%%%%%%%%%%%%%%%%
\section{Zukunft}
%% %%%%%%%%%%%%%%%%%%%%%%%%%%%%%%%%%%%%%%%%%%%%%%%%%%%%%%%
 \frame{
   \frametitle{Zukunft -- Was erwartet uns?}
    
   \begin{itemize}
   \item<1->Pad-Bots
   \item<2->Pad-Spam
   \item<3->Telko-Spam
   \item<4->Trolle !!!1!
   \end{itemize}
 }


%% %%%%%%%%%%%%%%%%%%%%%%%%%%%%%%%%%%%%%%%%%%%%%%%%%%%%%%%
%% \frame{
%%   \frametitle{}
    
%%   \begin{itemize}
%%   \item<1->
%%   \item<2->
%%   \item<3->
%%   \end{itemize}
%% }




%% %%%%%%%%%%%%%%%%%%%%%%%%%%%%%%%%%%%%%%%%%%%%%%%%%%%%%%%
%% \frame{
%% \frametitle{a contitional image}

%% \ifthenelse{\boolean{makeHandoutWithoutGraphics}}{}{
%%   \begin{textblock}{0}(1,1)
%%     \includegraphics<2>[scale=.3]{pics/some-image}
%%   \end{textblock}%
%% }



%%%%%%%%%%%%%%%%%%%%%%%%%%%%%%%%%%%%%%%%%%%%%%%%%%%%%%%
\ifthenelse{\boolean{makeHandoutWithoutGraphics}}{}{\frame{\Huge Noch Fragen?}}

\ifthenelse{\boolean{makeHandoutWithoutGraphics}}{}{\frame{
	\Huge Danke!

        \vspace{1cm}

	\normalsize
	Folien-Quellcode: \hfill \url{https://github.com/AnneGilles/piraten_tools}
             \begin{textblock}{0}(7.5,0)
                \includegraphics[scale=1]{img/github-ribbon}
             \end{textblock}%

        \vspace{1cm}

        Kontakt: christoph.scheid@piratenpartei-hessen.de

}}

\end{document}

